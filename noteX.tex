% main.tex
\documentclass[12pt,a4paper]{article}

% Essential packages
\usepackage[utf8]{inputenc}
\usepackage[T1]{fontenc}
\usepackage{amsmath}
\usepackage{amsfonts}
\usepackage{amssymb}
\usepackage{graphicx}
\usepackage[margin=2.5cm]{geometry}
\usepackage{hyperref}
\usepackage{fancyhdr}

% Additional useful packages
\usepackage{listings}           % For code snippets
\usepackage{xcolor}            % For colored text and boxes
\usepackage{tcolorbox}         % For nice looking boxes
\usepackage{enumitem}          % For better lists
\usepackage{tikz}              % For drawings and diagrams
\usepackage{mdframed}          % For framed theorems
\usepackage{mathtools}         % Enhanced math tools
\usepackage{tocloft}           % For customizing ToC
\usepackage{etoolbox}          % For patching commands

% Optional: Set up the directory where your images are stored
\graphicspath{{images/}}  % This means your images are in an "images" folder

% Document info (to be modified by user)
\newcommand{\courseTitle}{Course Title}
\newcommand{\authorName}{Author Name}
\newcommand{\courseCode}{CODE101}
\newcommand{\semester}{Fall 2024}
\newcommand{\noteDate}{\today}

% Custom counters for ToC
\newcounter{definitioncounter}[section]
\newcounter{examplecounter}[section]
\newcounter{listingcounter}[section]

% Custom environments for theorems, definitions, etc. with ToC entries
\newtheorem{theorem}{Theorem}[section]

\newenvironment{definition}[1]
{\refstepcounter{definitioncounter}
\addcontentsline{toc}{subsection}{Definition \thedefinitioncounter. #1}
\par\medskip\noindent\textbf{Definition \thedefinitioncounter.}}
{\par\medskip}

\newenvironment{example}
{\refstepcounter{examplecounter}
\addcontentsline{toc}{subsection}{Example \theexamplecounter}
\par\medskip\noindent\textbf{Example \theexamplecounter.}}
{\par\medskip}

% Make listings appear in ToC
\makeatletter
\newcommand{\listingscaption}[1]{%
  \refstepcounter{listingcounter}%
  \addcontentsline{toc}{subsection}{Listing \thelistingcounter: #1}%
  \lstset{caption=#1}%
}
\makeatother

% Custom box for important notes
\newtcolorbox{important}[1]{
    colback=red!5!white,
    colframe=red!75!black,
    title=\textbf{#1}
}

% Header and footer setup
\pagestyle{fancy}
\fancyhf{}
\rhead{\courseTitle}
\lhead{\authorName}
\rfoot{Page \thepage}

\begin{document}

% Title page
\begin{titlepage}
    \centering
    \vspace*{2cm}
    {\huge\bfseries \courseTitle\par}
    \vspace{1cm}
    {\Large \courseCode, \semester\par}
    \vspace{2cm}
    {\Large\itshape Lecture Notes\par}
    \vspace{3cm}
    {\Large\bfseries \authorName\par}
    \vfill
    {\large \noteDate\par}
\end{titlepage}

% Table of contents
\tableofcontents
\newpage

% Main content starts here
\section{Introduction}
Your notes start here...

\begin{important}{Important Note}
This is an important note that will be highlighted in a red box.
\end{important}

\section{Topic 1}
Content for topic 1...

\begin{definition}{Term to be defined}
This is how you can define important concepts.
\end{definition}

\begin{example}
This shows how to present examples clearly.
\end{example}

\section{Topic 2}
Content for topic 2...

% Code example
\begin{lstlisting}[language=Python, caption=Example Code]
def hello_world():
    print("Hello, World!")
\end{lstlisting}

\end{document}
